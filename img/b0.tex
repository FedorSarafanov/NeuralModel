\documentclass[tikz]{standalone}

\usepackage[T2A]{fontenc}
\usepackage[utf8x]{inputenc}
\usepackage{cmap,pgfplots,pgfplotstable}

\begin{document}

\pgfplotstableread{ 				% здесь нужно забивать данные
b0	n
1	13
2	10
3	8
4	8
5	8
6	8
7	8
8	8
9	8
10	8
11	8
12	8
}\mytable

	\begin{tikzpicture}
		\begin{axis}[
			%%%%%%%%%%%%%%%%%%%%%%%%%%%%% НАСТРОЙКИ ГРАФИКА %%%%%%%%%%%%%%%%%%%%%%%%%%%%%
			height=7.5cm,
			scale=1.25,
			grid=both, 				% включаем отображение сетки на графике
			%
			ylabel={$n$, спайков}, 			% подпись оси Y
			xlabel={$b_0$}, 			% подпись оси X
			%
			major grid style={
				line width=0.5pt, 	% толщина основных линий сетки
				draw=black!50 		% цвет основных линий сетки: 50% черного (80% белого) 
			},
			%
			minor grid style={
				line width=0.5pt, 	% толщина промежуточных линий сетки
				draw=black!20		% цвет промежуточных линий сетки
			},
			%
			minor tick num=4,		% количество промежуточных линий между основными
			%
 			ticklabel style={
 				scale=0.95			% уменьшим размер подписей меток на осях
 			},    
 			%
	    	axis lines=middle, 		% выравнивание оси y:  middle (в нуле)|left|right
	    	%
			ymin = 7,				% максимумы и минимумы осей на графике
			% ymin = 7,	
			xmax = 3,
			ymax = 14,
			xmin=0,
			%
			xtick distance=1,		% расстояние между метками по оси X
			ytick distance=1,		% расстояние между метками по оси Y
			%
			unit vector ratio = 1 1,% масштаб 1:1 осей X и Y
			%
			x label style={
				at={(axis description cs:1.07,0)},
				anchor=center,		% расположение метки ровно в точке (1.1,0)
				rotate=360,			% вообще метку еще можно повернуть)
				black				% цвет метки
			},
    		y label style={
    			at={(axis description cs:0,1.1)},
    			anchor=center,		% расположение метки ровно в точке (0,1.1)
    			black				% цвет метки
    		},			
			%							
			%%%%%%%%%%%%%%%%%%%%%%%%%%%%%%%%%%%%%%%%%%%%%%%%%%%%%%%%%%%%%%%%%%%%%%%%%%%%%
		]

		% Добавляем график на рисунок
		\addplot[
			blue!80,					% цвет линии графика
			% smooth,					% включает "сглаживание" линии графика
			line width=0.8pt, 	    % толщина основных линии графика
			mark=*,					% точки на графике: none (нету) или * (точка), 
			mark size=1pt,			% размер точки на графике
			% mark color=l
			% only marks, 			% если расскоментить, будет рисоваться только точки без линии графика
		] table {\mytable};
        
        % Легенда для этого
       \legend{}
       		\end{axis}
	\end{tikzpicture}	
\end{document}