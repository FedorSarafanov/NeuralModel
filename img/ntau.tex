\documentclass[tikz]{standalone}

\usepackage[T2A,T1]{fontenc}
\usepackage[utf8x]{inputenc}
\usepackage[english, russian]{babel}
\usepackage{cmap,pgfplots,pgfplotstable,amssymb,amsfonts,amsmath,amsthm}
\pgfplotsset{compat=newest}
\pgfplotstableread{ 				% здесь нужно забивать данные
x	y
2	9
4	8
6   13
}\mytable

\begin{document}
	\begin{tikzpicture}
		\begin{axis}[
			%%%%%%%%%%%%%%%%%%%%%%%%%%%%% НАСТРОЙКИ ГРАФИКА %%%%%%%%%%%%%%%%%%%%%%%%%%%%%
			% height=5cm,				% размер графика. Для презентации оптимальна высота 5см
			%
			% scale=1.25,				% масштабирование графика
			grid=both, 				% включаем отображение сетки на графике
			%
			major grid style={
				line width=0.5pt, 	% толщина основных линий сетки
				draw=black!50 		% цвет основных линий сетки: 50% черного (80% белого) 
			},
			%
			minor grid style={
				line width=0.5pt, 	% толщина промежуточных линий сетки
				draw=black!20		% цвет промежуточных линий сетки
			},
			%
			minor x tick num=4,	% количество промежуточных линий между основными
			minor y tick num=4,
			%
 			ticklabel style={
 				scale=0.95			% уменьшим размер подписей меток на осях
 			},    
 			%
			extra x ticks={},	% дополнительные метки на осях
			extra x tick labels={	% можно указать специальные подписи к ним
					{:)}
				}, 				
			extra tick style={		% свойства дополнительных меток
				% major grid style=red,% линия через метку - красная
        		% tick align=outside,	% положение дополнительной метки в стороне от основных
        		% tick style=red, 	% цвет черточки метки -- красный
        		% magenta,			% цвет текста метки - розовый
				% scale=1,
			},		
			tick style={		% свойства дополнительных меток
				scale=0.8,
			},			
			%
	    	axis lines=middle, 		% выравнивание оси y:  middle (в нуле)|left|right
	    	%
			xmin = 	0,			% максимумы и минимумы осей на графике
			ymin =  7,	
			% xmax =	4,
			ymax =	14,
			%
			xtick distance=1,		% расстояние между метками по оси X
			ytick distance=1,		% расстояние между метками по оси Y
			%
			unit vector ratio = 1 1,% масштаб 3:1 осей X и Y,
			%
			extra x ticks={0,2,4,6},		% свои метки по x
			xtick={0,1,...,10},		% свои метки по x
			xticklabels={},
			extra x tick labels={{},{$C_7$},{$C_7+C_8$},{$C_7+C_9$},{}},		
			%
			ylabel={$n$}, 			% подпись оси Y
			xlabel={$C$},			% подпись оси X
			%
			% Cвойства меток осей и их положение. В системе координат "axis description cs" 
			% координаты (0,0) -- левый нижний угол рисунка, (1,1) -- правый верхний
			x label style={
				at={(axis description cs:1.06,0)},
				anchor=center,		% расположение метки ровно в точке (1.1,0)
				rotate=360,			% вообще метку еще можно повернуть)
				black				% цвет метки
			},
    		y label style={
    			at={(axis description cs:0,1.1)},
    			anchor=center,		% расположение метки ровно в точке (0,1.1)
    			black				% цвет метки
    		},			
			%			
			%%%%%%%%%%%%%%%%%%%%%%%%%%%%%%%%%%%%%%%%%%%%%%%%%%%%%%%%%%%%%%%%%%%%%%%%%%%%%
		]

		% Добавляем график на рисунок
		\addplot[
			blue!80,					% цвет линии графика
			% smooth,					% включает "сглаживание" линии графика
			line width=0.8pt, 	    % толщина основных линии графика
			mark=*,					% точки на графике: none (нету) или * (точка), 
			mark size=1pt,			% размер точки на графике
			% only marks, 			% если расскоментить, будет рисоваться только точки без линии графика
		] table {\mytable};
        
        % Легенда для этого
        % \legend{}	

        % график функции x^2, x от 0 до 2, построить по 100 точкам
        % \addplot [samples=100, domain=0:2, red] {x^2};

		\end{axis}
	\end{tikzpicture}	
\end{document}