\input{text/preambule.tex}
\newcommand{\sq}[1]{\tikz{\draw[draw=#1,fill=#1] (0,0) rectangle (0.7em,0.7em);}}
\begin{document}  

%%%%%%%%%%%%%%%%%%%%%%%%%%%%%%%%%%%%%%%%%%%%%%%%%%%%%%%%%%%%%
\begin{frame}[plain]
	\centering
	\vspace{2cm}
	\begin{beamercolorbox}[sep=8pt,center]{title}
		\bf\usebeamerfont{title}\inserttitle
	\end{beamercolorbox}
	\vspace{0.5cm}
	\normalsize \textbf{Работу выполнили:}\\
	\large
	\underline{Платонова М.В.}, %
	{Рогов М.А.,}
	Сарафанов Ф.Г. %
	% Геликонова В.Г. %
	\\ 
	\vspace{0.5cm}
	\normalsize{\textbf{Научный руководитель:}\\}
	\large{Щапин Д.С.}
	\vfill
	\small{Нижний Новгород -- 2018}
\end{frame}
%%%%%%%%%%%%%%%%%%%%%%%%%%%%%%%%%%%%%%%%%%%%%%%%%%%%%%%%%%%%%
% \begin{frame}[t]s
% 	\frametitle{Содержание}
% 	% \fontsize{6pt}{7.2}\selectfont
% 	\setbeamerfont{subsection in toc}{size=\tiny}
% 	\setbeamerfont{section in toc}{size=\tiny}
% 	\tableofcontents
% \end{frame}
%%%%%%%%%%%%%%%%%%%%%%%%%%%%%%%%%%%%%%%%%%%%%%%%%%%%%%%%%%%%%

\section{Введение}
\subsection{Цели работы}
\begin{frame}[t]
	\frametitle{Цели работы}
	% \textbf{Цели}\\
	\vfill
	\begin{spacing}{1}
		\begin{enumerate}
			\item Ознакомиться с моделью нейрона с подпороговыми колебаниями
			\item Изучить модельные уравнения и эквивалентную схему
			\item Создать компьютерную модель нейрона
			\item Рассмотреть физическую реализацию модели в виде электронной установки
			\item Получить экспериментальные данные с установки и сравнить с теоретическими
			% феноменологическая модель
		\end{enumerate}
	\end{spacing}
	\vfill
\end{frame}
%%%%%%%%%%%%%%%%%%%%%%%%%%%%%%%%%%%%%%%%%%%%%%%%%%%%%%%%%%%%%
% \subsection{Глоссарий}
% \begin{frame}[c]
% 	\frametitle{Введениe}
% 	\begin{spacing}{1}
% 		\begin{enumerate}
% 			\item \textbf{Нейрон} – клетка, главное рассматриваемое свойство которой -- способность к генерации импульса возбуждения
% 			\item \textbf{Подпороговые колебания} – колебания квазигармонического типа, спонтанно возникающие в нейроне
% 			\item \textbf{Спайк} – единичный импульс, генерируемый нейроном как отклик на внешнее воздействие выше порога возбуждения
% 			\item \textbf{Спайк-берст} – режим колебаний, при котором на одном высоком периоде генерируется больше одного спайка  
% 		\end{enumerate}
% 		\vspace{1em}
% 		Способность нейрона к подпороговым колебаниям и генерации импульсов отражена в рассматриваемой модели.
% 	\end{spacing}
% \end{frame}
%%%%%%%%%%%%%%%%%%%%%%%%%%%%%%%%%%%%%%%%%%%%%%%%%%%%%%%%%%%%%
\section{Теоретическая часть}
\subsection{Подпороговые колебания в нейронах}
\begin{frame}[c]
	\frametitle{Подпороговые колебания в нейронах}
	\begin{columns}
		\begin{column}{0.49\textwidth}
			\begin{figure}[h]
				\centering
				\includegraphics[]{img/img_1a}
				\caption[]{Колебания в нейронах ствола головного мозга}
			\end{figure}
		\end{column}
		\begin{column}{0.49\textwidth}
			\begin{figure}[h]
				\centering
				\includegraphics[]{img/img_1d}	
				\caption[]{Подпороговые колебания в нейронах коры головного мозга}
			\end{figure}
		\end{column}
	\end{columns}
\end{frame}
%%%%%%%%%%%%%%%%%%%%%%%%%%%%%%%%%%%%%%%%%%%%%%%%%%%%%%%%%%%%%
\subsection{Генератор Ван-дер-Поля: фазовый портрет}
\begin{frame}[t]
	\frametitle{Генератор Ван-дер-Поля: фазовый портрет}
	\vspace{-1.2em}
	% \vspace{1em}
		
	\begin{columns}
		\begin{column}{0.49\textwidth}
			\begin{figure}[h]
				\centering
				\includegraphics[scale=0.99]{img/img_2a}
				\caption{$\gamma<0$:\\ устойчивый фокус}
			\end{figure}
		\end{column}
		\begin{column}{0.49\textwidth}
			\begin{figure}[h]
				\centering
				\includegraphics[scale=0.99]{img/img_2b}
				\caption{$\gamma>0$:\\ предельный цикл}
			\end{figure}
		\end{column}
	\end{columns}	
	\vspace{0.5em}
	\begin{columns}[c]
		% \begin{column}{0.49\textwidth}
		% \centering
		% В данной системе реализуются подпороговые колебания:
		% % При $\gamma=0$ происходит бифуркация Андронова-Хопфа и рождается предельный цикл
		% \end{column}
		\begin{column}{0.99\textwidth}
	\begin{equation*}
		\left\{
		\begin{aligned}
			\diff{y}{t} & =\mu(\gamma-x^2)y-\omega^2x\\%; \qquad
			\diff{x}{t}  &=y, \quad \mu \ll 1
		\end{aligned}
		\right.
	\end{equation*}		
		\end{column}
	\end{columns}
\end{frame}
%%%%%%%%%%%%%%%%%%%%%%%%%%%%%%%%%%%%%%%%%%%%%%%%%%%%%%%%%%%%%
\subsection{Модель ФитцХью-Нагумо: фазовый портрет}
\begin{frame}[t]
	\frametitle{Модель ФитцХью-Нагумо: фазовый портрет}
	% Способность к генерации спайков реализуется с помощью модели ФХН. Это простейшая система, обладающая порогом возбуждения:
	% \vspace{1em}
	\begin{columns}[c]	
		\begin{column}{0.59\textwidth}
			\vspace{-2em}
			\begin{figure}[h]
				\centering
				\hspace{1em}\includegraphics[]{img/img_3a}
				% \caption{3-уровневая среда}
			\end{figure}
			\vspace{-2em}
			\begin{figure}[h]
				\centering
				\hspace{1em}\includegraphics[]{img/img_3b}
				% \caption{3-уровневая среда}
			\end{figure}
		\end{column}
		\begin{column}{0.39\textwidth}
			$
				\left\{
				\begin{aligned}
					\diff{u}{t} & =f(u)-v           \\
					\diff{v}{t} & =\varepsilon(u-I)
				\end{aligned}
				\right.
			$\\
			\vspace{1em}
			$I$ -- параметр порога возбуждения\\
			\vspace{1em}
			$f(u)$ -- кубическая функция\\
			\vspace{1em}
			$\varepsilon \ll 1$ -- малый параметр
		\end{column}
	\end{columns}
\end{frame}
%%%%%%%%%%%%%%%%%%%%%%%%%%%%%%%%%%%%%%%%%%%%%%%%%%%%%%%%%%%%%
\subsection{Компьютерная модель Некоркина}
\begin{frame}%[bg]
	\frametitle{Математическая модель нейрона}
		\vspace{-2em}
	\begin{figure}[h]
		\hspace{-2em}
		\includegraphics[]{img/berst_matlab}
		% \caption{}
	\end{figure}
	
	\begin{columns}[t]
		\begin{column}{0.60\textwidth}
			\vspace{-3em}
			\begin{equation*}
				\left\{
				\begin{aligned}
					\varepsilon_1\diff{u}{t} & =f(u)-v-d\cdot x\\
					\diff{v}{t} & =\varepsilon_2(u-I)\\
					\diff{x}{t} & =y\\
					\diff{y}{t} & =\mu(\Gamma(u,I)-x^2)y-\Omega^2(u,I)\cdot x
				\end{aligned}
				\right.
			\end{equation*}
		\end{column}
		\begin{column}{0.2\textwidth}
			$a=0.1$\\
			$\varepsilon_1=0.001$\\
			$\varepsilon_2=1.5$\\
			$\gamma=0.21$\\
			$\omega=1$\\
		\end{column}
		\begin{column}{0.19\textwidth}
			$\alpha=5$\\
			$\beta=10$\\
			$I=-0.09$\\
			$d=0.85$\\
		\end{column}		
	\end{columns}
\end{frame}
%%%%%%%%%%%%%%%%%%%%%%%%%%%%%%%%%%%%%%%%%%%%%%%%%%%%%%%%%%%%%
\subsection{Характерные режимы модели}
\begin{frame}%[bg]
	\frametitle{Характерные режимы модели}
		% \vspace{-2em}
	\begin{figure}[h]
		\hspace{0em}
		\includegraphics[]{img/img_4}
		% \caption{}
	\end{figure}
	
	% \begin{columns}[t]
	% 	\begin{column}{0.60\textwidth}
	% 		\vspace{-3em}
	% 		\begin{equation*}
	% 			\left\{
	% 			\begin{aligned}
	% 				\varepsilon_1\diff{u}{t} & =f(u)-v-d\cdot x\\
	% 				\diff{v}{t} & =\varepsilon_2(u-I)\\
	% 				\diff{x}{t} & =y\\
	% 				\diff{y}{t} & =\mu(\Gamma(u,I)-x^2)y-\Omega^2(u,I)\cdot x
	% 			\end{aligned}
	% 			\right.
	% 		\end{equation*}
	% 	\end{column}
	% 	\begin{column}{0.2\textwidth}
	% 		$a=0.1$\\
	% 		$\varepsilon_1=0.001$\\
	% 		$\varepsilon_2=1.5$\\
	% 		$\gamma=0.21$\\
	% 		$\omega=1$\\
	% 	\end{column}
	% 	\begin{column}{0.19\textwidth}
	% 		$\alpha=5$\\
	% 		$\beta=10$\\
	% 		$I=-0.09$\\
	% 		$d=0.85$\\
	% 	\end{column}		
	% \end{columns}
\end{frame}
%%%%%%%%%%%%%%%%%%%%%%%%%%%%%%%%%%%%%%%%%%%%%%%%%%%%%%%%%%%%%
\subsection{Схема экспериментальной установки}
\begin{frame}[t]%[bg]
	\frametitle{Схема экспериментальной установки}
	\vspace{-1em}
	\begin{figure}[h]
		% \hspace{-2em}
		\includegraphics[width=0.86\linewidth]{img/img_6}
		% \caption{}
	\end{figure}
	\vspace{-0.5em}
	\begin{columns}
		\begin{column}{0.49\textwidth}%\centering
			\sq{black!90} -- генератор Ван-дер-Поля\\
			\sq{red} -- эмиттерный повторитель
		\end{column}
		\begin{column}{0.49\textwidth}%\centering
			\sq{blue} -- эмиттерный повторитель\\
			\sq{violet} -- блокинг-генератор
		\end{column}
	\end{columns}		
\end{frame}
%%%%%%%%%%%%%%%%%%%%%%%%%%%%%%%%%%%%%%%%%%%%%%%%%%%%%%%%%%%%%
\section{Результаты эксперимента}
\subsection{Подпороговые колебания}
\begin{frame}%[bg]
	\frametitle{Подпороговые колебания}
	\begin{figure}[h]
		\hspace{-2em}
		\includegraphics[]{img/osci}
		% \caption{}
	\end{figure}
\end{frame}
%%%%%%%%%%%%%%%%%%%%%%%%%%%%%%%%%%%%%%%%%%%%%%%%%%%%%%%%%%%%%
\subsection{Один спайк на несколько периодов}
\begin{frame}%[bg]
	\frametitle{Один спайк на несколько периодов}
	\begin{figure}[h]
		\hspace{-2em}
		\includegraphics[]{img/spike}
		% \caption{}
	\end{figure}
\end{frame}
%%%%%%%%%%%%%%%%%%%%%%%%%%%%%%%%%%%%%%%%%%%%%%%%%%%%%%%%%%%%%
\subsection{Один спайк на период}
\begin{frame}%[bg]
	\frametitle{Один спайк на период}
	\begin{figure}[h]
		\hspace{-2em}
		\includegraphics[]{img/onespike}
		% \caption{}
	\end{figure}
\end{frame}
%%%%%%%%%%%%%%%%%%%%%%%%%%%%%%%%%%%%%%%%%%%%%%%%%%%%%%%%%%%%%
\subsection{Два спайка на периоде}
\begin{frame}%[bg]
	\frametitle{Два спайка на периоде}
	\begin{figure}[h]
		\hspace{-2em}
		\includegraphics[]{img/twospike}
		% \caption{}
	\end{figure}
\end{frame}
%%%%%%%%%%%%%%%%%%%%%%%%%%%%%%%%%%%%%%%%%%%%%%%%%%%%%%%%%%%%%
\subsection{Спайк-берст}
\begin{frame}%[bg]
	\frametitle{Спайк-берст}
	\begin{figure}[h]
		\hspace{-1.3em}
		\centering
		\includegraphics[]{img/berst}
		\vspace{-1.5em}
		\caption{Эксперимент}
	\end{figure}
	\begin{figure}[h]
	% \vspace{-1em}
		\hspace{-2em}
		\centering
		\includegraphics[]{img/berst_matlab}
		% \vspace{-1.5em}
		\caption{Компьютерная модель}
	\end{figure}	
\end{frame}
%%%%%%%%%%%%%%%%%%%%%%%%%%%%%%%%%%%%%%%%%%%%%%%%%%%%%%%%%%%%%
\subsection{Влияние потенциала запирания $b_0$}
\begin{frame}%[bg]
	\frametitle{Влияние потенциала запирания $b_0$}
	\begin{figure}[h]
		% \hspace{2em}
		\includegraphics[]{img/b0}
		\caption{Зависимость количества спайков $n$ от величины \\потенциала запирания $b_0$ за время 50 мс }
	\end{figure}
\end{frame}
%%%%%%%%%%%%%%%%%%%%%%%%%%%%%%%%%%%%%%%%%%%%%%%%%%%%%%%%%%%%%
\subsection{Влияние времени релаксации на частоту спайков}
\begin{frame}%[bg]
	\frametitle{Влияние времени релаксации на частоту спайков}
	\begin{figure}[h]
		% \hspace{2em}
		\includegraphics[]{img/ntau2}
		\caption{Зависимость количества спайков $n$ от времени релаксации $\tau \sim C$ -- емкости конденсатора в цепи блокинг-генератора за время 50 мс }
	\end{figure}
\end{frame}

%%%%%%%%%%%%%%%%%%%%%%%%%%%%%%%%%%%%%%%%%%%%%%%%%%%%%%%%%%%%%
\section{Заключение}
\subsection{Выводы}
\begin{frame}
	\frametitle{Выводы}
	\begin{enumerate}
		\item Осуществлено знакомство с моделью нейрона с подпороговыми колебаниями
		\item Качественно исследованы уравнения, соответствующие квазигармоническим колебаниям и порогу возбуждения
		\item Получены подпороговые колебания, спайк и спайк-берст режим на экспериментальной установке
		\item Реализована компьютерная модель системы, на которой был получен спайк-берст режим, показано качественное соответствие режиму, полученному эксперементально
	\end{enumerate}
\end{frame}
%%%%%%%%%%%%%%%%%%%%%%%%%%%%%%%%%%%%%%%%%%%%%%%%%%%%%%%%%%
\subsection{Спасибо за внимание}
\begin{frame}[plain]
	\vspace{4cm}
	\begin{center}
		\Huge
		Спасибо за внимание!
	\end{center}
	\vspace{2.5cm}
	\begin{center}
		\color{black!60!white}
		Презентация подготовлена в издательской \\
		системе LaTeX с использованием пакетов \\
		PGF/TikZ и Beamer
	\end{center}
\end{frame}
\end{document}